\documentclass[journal]{IEEEtran}

\usepackage{palatino,epsfig,latexsym,cite,graphicx,amsmath,amssymb,amsfonts,multirow,booktabs,color,soul}
\usepackage{algorithmic,url}
\usepackage{float}
\usepackage{threeparttable}
\usepackage{amsmath}
\usepackage[linesnumbered,ruled,vlined]{algorithm2e}
\def\MR2{\multirow{2}[2]{*}}
\def\Hei{\hl}
\definecolor{hl}{rgb}{0.75,0.75,0.75}
\newcommand{\XG}[1]{\textcolor[rgb]{0.00,0.00,1.00}{#1}}
\newcommand{\TODO}[1]{\textcolor[rgb]{1.00,0.40,0.22}{#1}}
\sethlcolor{hl}
\usepackage{makecell}
\usepackage[caption=false,font=footnotesize]{subfig}
\renewcommand{\algorithmicrequire}{ \textbf{Input:}}      %Use Input in the format of Algorithm
\renewcommand{\algorithmicensure}{ \textbf{Output:}}     %Use Output in the format of Algorithm
\DeclareMathOperator*{\argmin}{argmin}

%\setlength{\abovecaptionskip}{-5pt}
\setlength{\floatsep}{12pt  minus 4pt}
\setlength{\textfloatsep}{13pt minus 4pt}
\renewcommand{\textfraction}{0.1}
\renewcommand{\topfraction}{0.9}
\setlength\arraycolsep{2pt}

\begin{document}

\title{{Adaptation Operator Selection with Reinforcement Learning for Multi-objective Evolutionary}
  \thanks{Thanks}}
\author{
  authors,
}% <-this % stops a space

% \markboth{IEEE Transactions on Cybernetics,~Vol.~, No.~, month~year}
% {Tian \MakeLowercase{\textit{et al.}}: SOLVING LMOPS WITH SPARSE OPTIMAL SOLUTIONS VIA UNSUPERVISED NNS}

\IEEEpubid{0000--0000/00\$00.00~\copyright~0000 IEEE}

\maketitle

\begin{abstract}
  Evolutionary Algorithms (EAs) are stochastic optimization algorithms which are greatly influenced by operators, different reproduction operators have various search range.
  Adaptive operators selection (AOS) provides the on-line control of the evolutionary algorithms that which operator should be chosen for the current individual.
  This paper proposes an AOS strategy based on Double Q-learning (DDQN), a deep Reinforcement Learning (RL) algorithm in discrete action spaces, to detect the operator should be applied.
  The two main components of AOS, Credit Assignment and Operator Selection, correspond to Reward and Action of Markov Decision Process (MDP) and the State of MDP can be represented by the features of the current individual. According to the above rules, the operator selection problem is modeled as MDP and can be solved by reinforcement learning. In the evolutionary process, RL will select an operator for each individual and update the behavioral strategy according to the effect of this selection. Experimental results on \TODO{supplement experimental instructions}.
\end{abstract}

\begin{IEEEkeywords}
  Multi-objective optimization, Reinforcement Learning, Credit assignment
\end{IEEEkeywords}

\section{Introduction}
% 介绍多目标优化
\IEEEPARstart{M}{ulti}-objective evolutionary algorithms have attracted great attention over the past few decades and achieved outstanding performance in solving various kinds of multi-objective optimization problems (MOPs) \cite{fialho2010adaptive,tian2019evolutionary}. In general, the MOPs can be defined as:

\begin{equation}
  \begin{array}{ll}
    \text { Minimize: }   & F=\left\{f_{1}(x), \ldots, f_{M}(x)\right\} \\
    \text { Subject to: } & x \in \Omega
  \end{array}
\end{equation}

% 介绍多目标进化算法
Where $x$ is a candidate solution from the decision space $\Omega \subseteq \mathbb{R}^{n}$ and $M$ is the number of objectives. $F$ consists of $m$ objective functions. Since the objectives of a MOP are usually conflicting with each other, it is unlikely that optimality is met simultaneously by a single feasible solution on all objectives \cite{BAZGAN201341}; instead, a number of solutions are optimal for different objectives, known as Pareto optimal solutions. All the Pareto optimal solutions constitute the Pareto set and the set of all Pareto optimal objective vectors is the Pareto front \cite{deb2001multi}.

% 有很多进化算法,很多算子,不同的算子适用于不同的问题,有不同的效果,引出算子选择
EAs are stochastic optimization algorithms motivated by Darwinian evolutionary theory, which are composed of \textit{evolutionary operator} and \textit{environmental selection}.
Due to their population-based characteristics, they are skilled in solving MOPs, during the last decades, a number of algorithms have been proposed, e.g., indicator-based evolutionary algorithm (IBEA) \cite{IBEA}, multiobjective evolutionary algorithm based on decomposition (MOEA/D)



\IEEEpubidadjcol

The second paragraph...
\section{Related Work}

\subsection{Subsection Name ... }
yihang

ling hang

\subsection{the second subsection name...}
test
\section{The Proposed Algorithm}
test
\subsection{Framework of MOEA/PSL}
test
\subsection{Pareto Optimal Subspace Learning and Offspring Generation in MOEA/PSL}

\subsection{Parameter Adaptation Strategy in MOEA/PSL}

\subsection{Computational Complexity of MOEA/PSL}


\section{Empirical Studies}


\subsection{Comparative Algorithms}

\subsection{Test Problems}

\subsection{Results on Benchmark Problems}

\subsection{Results on Real-World Problems}

\subsection{Effectiveness of the Two Neural Networks in MOEA/PSL}\label{sec:twoNNs}

\subsection{Effectiveness of the Parameter Adaptation Strategy in MOEA/PSL}

\section{Conclusions}


 (e.g., deep neural network training)

\ifCLASSOPTIONcaptionsoff
  \newpage
\fi

\bibliographystyle{IEEEtran}
\bibliography{references}

% \iffalse
%   \begin{IEEEbiography}[{\includegraphics[width=1in,height=1.25in,clip,keepaspectratio]{XingyiZhang.eps}}]{Xingyi Zhang}
%     received the B.Sc. from Fuyang Normal College in 2003, and the M.Sc. in 2006 and Ph.D. in 2009
%     both from Huazhong University of Science and Technology. Currently, he is
%     an associate professor in the School of Computer Science and Technology, Anhui University. His main research interests
%     include unconventional models and algorithms of computation, multi-objective optimization and membrane computing.
%   \end{IEEEbiography}
% \fi

\end{document}